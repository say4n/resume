%-------------------------------------------------------------------------------
%	PACKAGES AND OTHER DOCUMENT CONFIGURATIONS
%-------------------------------------------------------------------------------

%%%%%%%%%%%%%%%%%%%%%%%%%%%%%%%%%%%%%%%%%%%%%%%%%%%%%%%%%%%%%%%%%%%%%%%%%%%%%%%%
% You can have multiple style options the legal options ones are:
%
%   centered:	the name and address are centered at the top of the page
%				(default)
%
%   line:		the name is the left with a horizontal line then the address to
%				the right
%
%   overlapped:	the section titles overlap the body text (default)
%
%   margin:		the section titles are to the left of the body text
%
%   11pt:		use 11 point fonts instead of 10 point fonts
%
%   12pt:		use 12 point fonts instead of 10 point fonts
%
%%%%%%%%%%%%%%%%%%%%%%%%%%%%%%%%%%%%%%%%%%%%%%%%%%%%%%%%%%%%%%%%%%%%%%%%%%%%%%%%

\documentclass[mm]{simple_style}

% Default font is the helvetica postscript font
\usepackage{helvet}
\usepackage{hyperref}
\usepackage{url}
\usepackage{xcolor}
\hypersetup {
    colorlinks=true,
    linkcolor=colorlink,
    filecolor=magenta,
    urlcolor=colorlink,
    pdftitle={Sayan Goswami CV},
}
\urlstyle{same}
\usepackage[left=0.7in, right=2in, top=0.9in]{geometry}

\definecolor{mgray}{gray}{0.8}

% Increase text height
\textheight=700pt

\begin{document}

%-------------------------------------------------------------------------------
%	NAME AND ADDRESS SECTION
%-------------------------------------------------------------------------------
\name{Sayan Goswami}
% \qualification{}
\email{sayan.goswami.106@gmail.com}
\website{https://sayan.page}{\url{https://sayan.page}}
\github{https://github.com/say4n}{\url{https://github.com/say4n}}
\phone{+91-833-700-5364}
\address{Kolkata, India}

%-------------------------------------------------------------------------------

\begin{resume}

%-------------------------------------------------------------------------------
%	EDUCATION SECTION
%-------------------------------------------------------------------------------
\section{Education}
\cusemph{Universitat Pompeu Fabra}, Barcelona, Spain \timeline{2020 - 2021}\\
{\it Master of Science}, Intelligent Interactive Systems\\

\vspace{-4ex}

\cusemph{Jadavpur University}, Kolkata, India \timeline{2016 - 2020}\\
{\it Bachelor of Engineering}, Electronics \& Telecommunication Engineering\\
\sectionline
%-------------------------------------------------------------------------------


%-------------------------------------------------------------------------------
%	EXPERIENCE SECTION
%-------------------------------------------------------------------------------

\section{Experience}

\cusemph{SDE}, Amazon, Edinburgh  \timeline{Jul '22 - Present}\\
SDE at Amazon Advertising.\\

\vspace{-4.5ex}

\cusemph{SRE}, Sea AI Lab, Singapore \timeline{Apr '22 - Jun '22}\\
DevOps, MLOps -- created monitoring tools for internal HPC cluster, established incident management practices, migrated actively used container registry with zero downtime.\\

\vspace{-4.5ex}

\cusemph{Research Programmer}, Quantum Information Group, UAB, Barcelona \timeline{Sep '21 - Mar '22}\\
Research on applying ML methods to optimize continuos variable quantum computing circuits.\\

\vspace{-4.5ex}

\cusemph{Mentor, Google Summer of Code}, mlpack.org \timeline{May '21 - Aug '21}\\
Guiding mentees on multi-objective optimization methods for mlpack, a C++ ML library.\\

\vspace{-4.5ex}

\cusemph{Research Fellow, AI Research Institute (IIIA-CSIC)}, Barcelona \timeline{Nov '20 - Aug '21}\\
Applying RL to model realistic behaviour of NPCs in simulation environments.\\

\vspace{-4.5ex}

\cusemph{Core Contributor, mlpack.org}, Remote \timeline{Sep '20 - Present}\\
Maintaining the popular C++ based open-source machine learning framework mlpack.\\

\vspace{-4.5ex}

\cusemph{Developer Associate, Samsung R\&D Institute}, Bangalore \timeline{May '19 - Jul '19}\\
Significantly decreased latency, increased throughput over QUIC protocol for wireless use.\\

\vspace{-4.5ex}

\cusemph{Mentor, Deep Reinforcement Learning Nanodegree}, Udacity \timeline{Jul '19 - Jan '20}\\
Guided students taking Udacity's Deep RL ``Nanodegree'', weekly meetings, coursework.\\

\vspace{-4.5ex}

\cusemph{Research Fellow, Vison \& Image Processing Lab}, IIT Bombay \timeline{May '18 - Jul '18}\\
Worked on deep learning (CNNs, GANs) based CV methods for image co-segmentation with \supervisor{Prof. Subhasis Chaudhuri}.\\

\vspace{-4.5ex}

\cusemph{Research Assistant, AI Lab}, Javapur University \timeline{Aug '18 - May '20}\\
Worked on multi-agent RL, game theory, neural control interface \& algorithms research with \supervisor{Prof. Amit Konar}.\\

\vspace{-4.5ex}

\cusemph{Community Mentor, Convolutional NN Course}, Coursera \timeline{Jan '18 - Jul '19}\\
Guided students taking Andrew Ng's CNN course on Coursera via community forums.\\

\vspace{-4.5ex}

\cusemph{Research Assistant, NLP Lab}, Javapur University \timeline{Jul '17 - Apr '19}\\
Worked on NLP methods for abstractive text summarization with \supervisor{Prof. Sudip K. Naskar}.

\vspace{-2ex}
\sectionline
%-------------------------------------------------------------------------------


%-------------------------------------------------------------------------------
%       AWARDS & ACHIEVEMENTS
%-------------------------------------------------------------------------------
\section{Awards \& Achievements}
Awarded \cusemph{JAE Intro ICU Fellowship} by the Spanish National Research Council (CSIC) in 2020.\\
Awarded \cusemph{Summer Research Fellowship} by the Indian Academy of Sciences in 2018.\\
\cusemph{National Finalist} at Automate for the Bank hackathon organised by State Bank of India in 2018.\\
Secured a \cusemph{National Rank of 228} in WBJEE amongst 150,000 candidates in 2016.\\
Secured a \cusemph{National Rank of 26, Zonal Rank of 2} in National Cyber Olympiad in 2016.\\
\cusemph{Regional Finalist} at TCS IT Wiz Quiz (top 3/100 teams) in 2015.

\vspace{-2ex}
\sectionline
%-------------------------------------------------------------------------------


%-------------------------------------------------------------------------------
%	KEY SKILLS SECTION
%-------------------------------------------------------------------------------
\section{Key\\Skills}
\cusemph{Programming}: Python, Golang, C++/C, Unix Scripting, Git, Tensorflow, Pytorch, Keras, MapReduce (Hadoop), MATLAB, Java/Kotlin (Android), Haskell, \LaTeX, Assembly (x86, MIPS), Verilog, SQL, HTML, React, Typescript, Redis, Django, Flask
\\

\vspace{-4ex}

\cusemph{Machine Learning \& Data Analysis}: Reinforcement Learning (Factored MDP, Bandits, Options Framework), Deep Learning (CNNs, RNNs, GANs), Machine Learning (SVM, KNN, Decision Trees, Bayes)

\vspace{-2ex}
\sectionline
%-------------------------------------------------------------------------------


\newpage
\sectionline
%-------------------------------------------------------------------------------
%      PUBLICATIONS
%-------------------------------------------------------------------------------

\section{Publications}
``Brain Signal Analysis for Mind Controlled Type-Writer Using a Deep Neural Network'' -- $5^{th}$ WiSPNET, 2020, Rohini Das, \cusemph{Sayan Goswami}, Sayantani Ghosh, Mousumi Laha, Chandrima Debnath and Amit Konar\\

\vspace{-4ex}

``Relationship between Nash Equilibria and Pareto Optimal Solutions for Games of Pure Coordination'' -- $10^{th}$ ICCCNT, 2019, Rohini Das, \cusemph{Sayan Goswami} and Amit Konar\\

\vspace{-4ex}

``Application of Deep Neural Network on Image Co-segmentation'' -- Indian Academy of Sciences SRF Report, 2018, \cusemph{Sayan Goswami} and Subhasis Chaudhuri

\vspace{-2ex}
\sectionline
%-------------------------------------------------------------------------------


%-------------------------------------------------------------------------------
%      PROJECTS
%-------------------------------------------------------------------------------
\section{Select\\Projects}

\href{https://github.com/say4n/dns.amplify}{\cusemph{dns.amplify}} -- A proof of concept implementation to understand DNS amplification based DDoS attacks.\\
\href{https://github.com/say4n/metal.compute}{\cusemph{metal.compute}} -- A C++ example showcasing the use of Apple's Metal API for general purpose GPU accelerated compute.\\
\href{https://github.com/say4n/bandit.rl}{\cusemph{bandit.rl}} -- A k-armed bandit test bed implementation for comparing various reinforcement learning algorithms.\\
\href{https://github.com/mlpack/ensmallen}{\cusemph{mlpack}} -- Implemented a framework for multi-objective optimization in the popular open-source C++ machine learning library mlpack.\\
\href{https://github.com/say4n/rtx.go}{\cusemph{rtx.go}} -- A brute force ray tracing implementation.\\
\href{https://github.com/say4n/eightyfive}{\cusemph{eightyfive}} -- An emulator for Intel’s 8085.\\
\href{https://github.com/say4n/ysh}{\cusemph{ysh}} -- An UNIX shell implementation.\\
\href{https://github.com/say4n/gobi}{\cusemph{gobi}} -- An in-memory database with a query language.\\
\href{https://github.com/say4n/infinity}{\cusemph{infinity}} -- A signed, arbitrary precision decimal arithmetic library for C++, dynamically linked at compile time.\\
\href{https://github.com/say4n/deepcosegmentation.pytorch}{\cusemph{Deep Co-segmentation}} – Deep object co-segmentation with deep convolutional neural networks using a siamese architecture.\\
\href{https://github.com/say4n/pytorch-segnet}{\cusemph{SegNet}} -- Semantic image segmentation using deep convolutional auto-encoders.\\
\href{https://github.com/say4n/flow}{\cusemph{flow}} -- Visualiser for control flow of arbitrary python code.\\
\href{https://github.com/say4n/fsmutil}{\cusemph{fsmutil}} -- A finite state machine generator for binary sequence detection.\\
\href{https://github.com/say4n/pyscuss}{\cusemph{Pyscuss}} – A real time messaging app, uses web sockets, non-persistent sessions.\\
\href{https://github.com/say4n/bfutil}{\cusemph{bfutil}} -- An interpreter for the BF language and a translator from BF to C with optimisations.\\
\href{https://github.com/say4n/LinkTo}{\cusemph{LinkTo}} -- An URL shortener with analytics dashboard, built using Flask framework, uses Redis as datastore.

\vspace{-2ex}
\sectionline
%-------------------------------------------------------------------------------

%-------------------------------------------------------------------------------
%      REFERENCES
%-------------------------------------------------------------------------------
\section{References}

Available on request.

\vspace{-2ex}
\sectionline
%-------------------------------------------------------------------------------

\vfill
\hfill \color{mgray}{Last updated on \today}


\end{resume}
\end{document}
